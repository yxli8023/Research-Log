\documentclass{project-log}
\SetMaintainer{第一作者身份}{第一作者姓名}{单位} % 主要作者的id是main
\CreateContributor{abc}{第二作者身份}{第二作者姓名}{单位} % 第一个变量是贡献者id,后面直接使用就可以标记是哪个人对应的内容
\SetProjectTitle{标题}
\SetProjectSubtitle{研究想法及科研日志}
\SetProjectHeaderName{科研进展}
\SetProjectSummary{文档简介} %文档介绍
\SetInstitutionLogo{figures/mcod_logo.png} % 文档logo
%\addbibresource{mybib.bib}% Set the bibliography file
%====================================================================
% 自定义命令
\newcommand{\eq}[1]{\begin{equation} #1 \end{equation}}
\newcommand{\eqa}[1]{\begin{equation}\bgin{aligned} #1 \end{aligned}\end{equation}}
%====================================================================
% Start the document
\begin{document}
%\pagestyle{plain}
\MakeFrontPage%  Front page
\setcounter{page}{1}%当前页为第一页
\pagenumbering{Roman}
\newpage
%--------------------------------------------------------
%\input{sec-2}
%--------------------------------------------------------
\newpage
\section{样式展示}
\setcounter{page}{1}%当前页为第一页
\pagenumbering{arabic}
\begin{Meeting}{会议日期}{main}%最后一个参数是作者id,用来标记这次会议是谁的想法
这里是一次会议讨论内容的记录,主导者第一个作者,用mian来作为他的id。在这次讨论中有一个重要任务先要完成
\hightodo{日期}{main}{这里是目前最优先要做的事情,这些事情都会被加入到最后的todo list 中 }
\end{Meeting}

\begin{Meeting}{会议日期}{abc}%最后一个参数是作者id,用来标记这次会议是谁的想法
这里是一次会议讨论内容的记录,主导者第一个作者,用abc来作为他的id。这次讨论中有一个次要任务需要完成
\lowtodo{日期}{abc}{这里是目前次优先要做的事情 ,这些事情都会被加入到最后的todo list 中}
\end{Meeting}

\begin{Note}{这是Note标题}
这里可以一些重要的笔记进行整理。
\end{Note}


\begin{Think}{这里是think 模块}
这个模块其实就是换了一个颜色,可以用来整理自己的一些想法。
\end{Think}




%----------------------------------------------------------------------------------------
\newpage
\appendix

%% References (bibliography)
%% Add the list of todos in the end
\section{Todo list}
\begin{Note}{}
\centering 这里总结了在讨论中遗留的问题以及之后需要解决的问题。
\end{Note}

\AddListOfTodos
%-----------------------------------------------------------------------------------------
%参考文献
\newpage
\bibliography{mybib}%加入参考文献
\bibliographystyle{unsrt}

\end{document}
